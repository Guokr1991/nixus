\documentclass[a4paper]{article}
%%% Local Variables: 
%%% mode: latex
%%% TeX-master: t
%%% End: 
\usepackage[utf8]{inputenc}
\usepackage[brazil]{babel}
\usepackage{graphicx}
\usepackage{hyperref}
\usepackage{listings}
\lstset{basicstyle=\scriptsize,frame=single,
  numbers=left,numberstyle=\tiny}

\def\TR{\textsuperscript{\textregistered}}
\def\TC{\textsuperscript{\textcopyright}}
\def\TM{\textsuperscript{\texttrademark}}

\begin{document}

\title{UTILIZAÇÃO DAS EXTENSÕES MULTIMÍDIA DOS PROCESSADORES INTEL\TR{}
  PARA REDUÇÃO DO NÚMERO DE CICLOS PARA A EXECUÇÃO DE PROGRAMAS
  HOLANDA}

\author{HOLANDA, Adriano de Jesus\footnote{Departamento de Computação
    e Matemática (DCM) da Faculdade de Filosofia, Ciências e Letras de
    Ribeirão Preto (FFCLRP) -- USP},\\ RUIZ, Evandro Eduardo Seron$^*$;\\
  CARNEIRO, Antônio Adilton~\footnote{Departamento de Física, FFCLRP
    -- USP}}

\date{11 de março de 2013}

\maketitle

\centerline{\scriptsize Submetido para publicação}.

\begin{abstract}
  A utilização das extensões multimídias com registradores que
  realizam a mesma operação sobre vários dados ao mesmo tempo (SIMD)
  dos atuais processadores podem reduzir o tempo de execução de
  programas que lidam com operações aritméticas sobre grande
  quantidade de dados. O objetivo deste trabalho foi quantificar o
  número de ciclos utilizados para o cálculo da correlação cruzada em
  duas dimensões para várias séries geradas e de diferentes tamanhos,
  usando a linguagem de programação C e as extensões para cálculo
  multimídia em Assembly para a codificação das instruções, compilação
  e execução. A comparação entre os resultados, usando o mesmo
  algoritmo e conjunto de dados, demonstrou que o programa em Assembly
  usando a extensão SIMD utilizou $38,37\%$ menos ciclos de
  processador que a mesma implementação escrita em C.

\end{abstract}
\noindent {\bf Palavras-chave:}
 Assembly. Processador. {\sc SIMD}. Correlação cruzada.

\section*{INTRODUÇÃO}

Mesmo com a disseminação dos processadores com arquitetura de 64 bits
substituindo os de 32 bits, as demandas por maior capacidade de
armazenamento temporário nos registradores e melhor eficiência na
operação sobre os dados armazenados vêm pressionando os fabricantes de
processadores a adotarem soluções que se desviam da arquitetura
normalmente empregada para construção dos processadores.  A principal
modificação é a inclusão de registradores de 128 e 256 bits para a
realização de vários cálculos de uma única vez, aumentando a
performance dos programas, que utilizam estes recursos, causado pela
redução no número de ciclos durante o processamento.  Estes recursos
que começaram a ser manufaturados pela Intel® permitem que múltiplos
dados sejam manipulados por uma única instrução (SIMD – {\em Single
Instruction Multiple Data}), e fazem parte da tecnologia de extensão
multimídia (MMX – {\em Multimedia Extension}) introduzida a partir do
Pentium II, para realizar operações complexas sobre números inteiros.
Uma extensão da tecnologia SIMD denominada SSE (Streaming SIMD
Extension) possui registradores de 128 bits para o armazenamento de
múltiplos dados. A Figura 1 mostra como dados do tipo inteiro de
diferentes tamanhos podem ser armazenados em um registrador
SSE~\cite{blum}.

\begin{figure}[h]
  \centering
  \includegraphics[scale=.4]{sse.png}
  \caption{ Possibilidade de divisões dos registradores XMM de 128-bit
    para armazenamento de números inteiros.  (Fonte:~\cite{ray})}
\label{fig:xmm128}
\end{figure}

Utilizando os registradores de 128-bit para armazenamento de 2 números
inteiros de 64 bits, que poderiam representar 2 pixels de um mapa de
bits de imagens RGB (Red x Green x Blue), a soma de 2 números dois a
dois pode ser realizada em 1 ciclo do processador, enquanto utilizando
o método tradicional seriam necessárias dois ciclos do processador
para a execução da operação. A Figura 2 ilustra como a operação em
dados múltiplos é utilizada para redução do número total de ciclos
para o processamento.

\begin{figure}[h]
  \centering
    \includegraphics[scale=.4]{sseop.png}
  \caption{Operação realizada em múltiplos dados sobre o mesmo registrador.}
  \label{fig:simd}
\end{figure}

A Figura 4 ilustra a execução de uma mesma operação em vários dados,
onde uma operação (OP) é realizada sobre os primeiros segmentos de
cada registrador e o resultado armazenado no primeiro segmento do
registrador destino, enquanto a mesma operação é realizada sobre os
segundos segmentos com o armazenamento ocorrendo no segundo segmento
do registrador destino. Este é o modo de operação das instruções
Assembly {\tt paddq} e {\tt pmuldq}, adição e multiplicação de quadwords
empacotados, utilizadas em nosso trabalho~\cite{intel}.  A utilização
destes recursos para a execução de operações aritméticas podem reduzir
o número de ciclos necessários para completar um tarefa, reduzindo o
tempo de execução do programa.  

\section*{OBJETIVO}
O objetivo deste trabalho é quantificar a diferença no tempo de
execução de instruções que utilizam a compilação na linguagem C sem o
uso das extensões SSE e com uso da extensão SSE com registradores de
128 bits e divididos em inteiros de 32 bits.

\section*{MATERIAL E MÉTODOS}

\subsection*{Codificação}

O algoritmo escolhido para os testes é utilizado para o cálculo de
correlação cruzada entre duas séries de valores igual tamanho, sua
fórmula original é mostrada na Equação~\ref{eq:XCORR}:

 \begin{equation}
     r_{xy}= \frac{ \sum (x_i-\overline{x})(y_i- \overline{y}) }{
  \sqrt{\sum (x_i-\overline{x})^2 \sum (y_i-\overline{y})^2}},
   \label{eq:XCORR}
 \end{equation}


\noindent onde a média dos valores é calculada por

\begin{equation}
\overline{x} = \frac{\sum{x_i}}{n}.
  \label{eq:mean}
\end{equation}

Porém, a Equação 1 tem complexidade $O(n^2)$, onde n é o número de
elementos na série, pois há a necessidade de realizar duas passagens
pelos dados, uma para calcular a média das séries x e y, e outra para
calcular o valor da correlação cruzada $r_{xy}$ entre as séries. A Equação
1 pode ser decomposta de modo a necessitar somente de uma passagem
pelos dados, possuindo complexidade $O(n)$ (Equação~\ref{eq:xcorr}):

\begin{equation}
r_{xy} = \frac{ n\sum x_iy_i-\sum x_i\sum y_i } { \sqrt{n\sum
    x_i^2-(\sum x_i)^2} \sqrt{n\sum y_i^2-(\sum y_i)^2} }.
  \label{eq:xcorr}
\end{equation}

A correlação cruzada é comumente utilizada em processamento de sinais
e imagens para analisar as similaridades entre as séries de dados que
podem ser valores de pixels ou intensidade do sinal em determinado
tempo.  A implementação da Equação 3 utilizando a linguagem C é
apresentada na Figura~\ref{fig:c-code} e foi sugerida por Seyfard (2012), e
modificada pelos autores para operar sobre números inteiros:

\begin{figure}[h]
  \centering

\begin{lstlisting}[language=C]
  double xcorr(long x[], long y[], n)
  {
    register long i;
    register long sum_x=0, sum_y=0, 
    sum_xx=0, sum_yy=0, sum_xy=0;
    for (i = 0; i < n ; i++) {
      register long xval = *(x+i);
      register long yval = *(y+i);
      sum_x += xval;
      sum_y += yval;
      sum_xx += xval * xval;
      sum_yy += yval * yval;
      sum_xy += xval * yval;
    }
    return ((double)(n*sum_xy-sum_x*sum_y))/
    (sqrt((n*sum_xx-sum_x*sum_x)* 
    (n*sum_yy-sum_y*sum_y)));
  }
\end{lstlisting}
  \caption{Código em linguagem C do cálculo da correlação cruzada entre duas séries de dado}
  \label{fig:c-code}
\end{figure}

O código implementado em Assembly para processadores Intel® x86 de 64
bits é mostrado na Figura~\ref{fig:asm-code}. O código também segue, em linhas gerais,
o descrito por Seyfarth (2012), com algumas modificações. A primeira é
utilizar somente números inteiros para as séries, pois normalmente os
valores de pixels na imagem são do tipo inteiro. A segunda modificação
é uma verificação no cálculo final da correlação para evitar divisões
por zero, que ; poderiam causar erro fatal ao final da execução do
programa.  1

\begin{figure}[h]
  \centering
  \begin{lstlisting}[]
    segment .data
one dq  1
    segment .text
    global xcorr_nat
xcorr_nat:
    xor r8,r8
    move rcx, rdx
    subpd xmm0, xmm0 ; zera sum_x
    movdqa xmm1, xmm0 ; zera sum_y
    movdqa xmm2, xmm0 ; zera sum_xx
    movdqa xmm3, xmm0 ; zera sum_yy
    movdqa xmm4, xmm0 ; zera sum_xy
    movdqa xmm8, xmm0 ; zera n
    movdqa xmm9, xmm0 ; zera variavel temporaria
.loop:                    ;; Main loop
    movdqa xmm5, [rdi+r8] ; atribui valor do proximo x da serie
    movdqa xmm6, [rsi+r8] ; atribui valor do proximo y da serie
    paddq xmm0, xmm5      ; sum_x
    paddq xmm1, xmm6      ; sum_y
    movdqa xmm7, xmm5     ; atribui x a um registrador temporario
    pmuldq xmm7, xmm6     ; x*y
    pmuldq xmm5, xmm5     ; x*x
    pmuldq xmm6, xmm6     ; y*y
    paddq xmm2, xmm5      ; sum_xx
    paddq xmm3, xmm6      ; sum_yy
    paddq xmm4, xmm7      ; sum_xy
    add r8, 16 
    sub rcx, 2
    jnz .loop
    haddpd xmm0, xmm0     ; compacta sum_x
    cvtdq2pd xmm0, xmm0   ; converte sum_x para ponto flutuante
    haddpd xmm1, xmm1     ; compacta sum_y
    cvtdq2pd xmm1, xmm1   ; converte sum_y para ponto flutuante
    haddpd xmm2, xmm2     ; compacta sum_xx
    cvtdq2pd xmm2, xmm2   ; converte sum_xx para ponto flutuante
    haddpd xmm3, xmm3     ; compacta sum_yy
    cvtdq2pd xmm3, xmm3   ; converte sum_yy para ponto flutuante
    haddpd xmm4, xmm4     ; compacta sum_xy
    cvtdq2pd xmm4, xmm4   ; converte sum_xy para ponto flutuante
    cvtsi2sd xmm8, rdx    ; armazena o valor de n em xmm8
    mulsd xmm4, xmm8      ; n*sum_xy
    movsd xmm9, xmm0      ; tmp := sum_x
    mulsd xmm9, xmm1      ; sum_x*sum_y
    subsd xmm4, xmm9      ; numerador := n*sum_xy - sum_x*sum_y
    mulsd xmm2, xmm8      ; n*sum_xx
    mulsd xmm0, xmm0      ; sum_x*sum_x
    subsd xmm2, xmm0      ; denominador0 := n*sum_xx - sum_x*sum_x
    sqrtsd xmm2, xmm2     ; denominador0 := sqrt(denominador0)
    mulsd xmm3, xmm8      ; n*sum_yy
    mulsd xmm1, xmm1      ; sum_y*sum_y
    subsd xmm3, xmm1      ; denominador1 := n*sum_yy - sum_y*sum_y
    sqrtsd xmm3, xmm3     ; denominador1 := sqrt(denominador1)
    mulsd xmm2, xmm3      ; denominador := denominador0 * denominador1
    movsd xmm9, xmm2      ; tmp := denominador
    comisd xmm4,xmm9      ; denominador  == numerador ?
    jz DEN_EQ_NUM         ; (denominador  == numerador) := verdadeiro
    divsd xmm4, xmm2      ; correlacao := numerador/denominador
    movsd xmm0, xmm4      ; retorna correlacao
    jz EXIT
    DEN_EQ_NUM: movsd xmm0, [one] ; correlacao := 1
    cvtdq2pd xmm0, xmm0  ; converte inteiro para ponto flutuante
EXIT:

\end{lstlisting}
  
\caption{Código em Assembly Intel® x86-64 para o cálculo da correlação
  cruzada entre duas séries de números inteiros.}
  \label{fig:asm-code}
\end{figure}

Na Figura~\ref{fig:asm-code} o código em Assembly Intel® x86-64 é
mostrado com o número da linha de cada instrução na primeira coluna,
os rótulos na segunda coluna, as operações na terceira, os operandos
na quarta e os comentários na quinta após o símbolo “;”.

Das linhas 15 à 29 estão as instruções para a realização do somatório
para o cálculo da correlação cruzada conforme mostrado na
Equação~\ref{eq:xcorr}. As instruções paddq e pmuldq executam a soma e
multiplicação, respectivamente, de quatro valores inteiros conforme
mostrado na Figura~\ref{fig:simd}. Das linhas 30 à 62, há instruções para a
preparação dos resultados das somas, convertendo-os para ponto
flutuante, para efetuar a divisão, e realizando cálculo final da
correlação. A linha 56 garante que se o numerador e denominador da
equação forem iguais a zero, ou seja a correlação seja igual a 1, a
divisão pelo denominador igual a zero, não seja realizada retornando o
valor 1 sem efetuar a divisão final.

Os códigos mostrados neste trabalho fazem parte um projeto de análise
elastográfica de imagens de ultrassom desenvolvido pelo Departamento
de Física da Faculdade de Filosofia, Ciências e Letras de Ribeirão
Preto/USP, e podem ser obtidos pelo site
\url{https://github.com/ajholanda/nixus}.  

\subsection*{Compilação e execução}

O programa em C foi compilado usando o gcc (GNU Compiler Collection,
\url{http://gcc.gnu.org/}) sem nenhuma flag adicional passada para o
compilador. O código em Assembly, conforme mostrado na Figura 2, foi
compilado usando o assembler Yasm (\url{http://yasm.tortall.net/}) com as
seguintes flags: ``{\tt -Worphan-labels -f elf64 -g dwarf2 -l corr.lst}''. Os
programas foram compilado e executados em um computador com
processador Intel® Core\TM{} 2 Duo com 2,93 GHz e 3MB de memória cache, e
memória DRAM com 3GB de capacidade.  O número de valores atribuídos
para as séries foram gerados a partir dos índices do arranjo contendo
os elementos com as seguintes atribuições:

\begin{lstlisting}[language=C,frame=noframe,numbers=right]
  x[i] = 1+i;
  y[i] = 1+2*i;
\end{lstlisting}

\noindent onde {\tt i} é o índice do arranjo onde está sendo efetuado
o loop.

Os cálculos de correlação entre as séries foram efetuados para os
seguintes números de elementos:

\begin{center}
\begin{tt}
	<1.000,1.000,9.000>, \\
	<10.000, 10.000, 90.000>, \\
	<1.000.000, 1.000.000, 9.000.000>, \\
	<10.000.000, 10.000.000, 90.000.000>,\\
\end{tt}
\end{center}

\noindent onde o primeiro valor da tupla indica valor inicial para o
loop, o segundo o passo e o terceiro o limite do loop.

\section*{RESULTADOS}

A Figura~\ref{fig:c-versus-asm} contêm a relação entre os número de
pontos das séries $x$ e $y$ e o número de ciclos utilizados para calcular
a correlação entre as séries utilizando o programa em C e Assembly.

\begin{figure}[h]
  \centering
  \includegraphics[scale=.525]{xcorr-bench.png}
  \caption{Número de ciclos utilizados pelo processador em função de
    número de elementos das séries geradas para o cálculo da
    correlação cruzada.}
  \label{fig:c-versus-asm}
\end{figure}

Em média houve uma redução de $38,37\%$ do número de ciclos utilizando
o código em Assembly quando comparado com o código em C. Para um
número menor de valores na série, a performance do código em Assembly
é maior devido ao fato da menor utilização da memória cache. Para o
número de elementos na série entre $10.000$ e $100.000$, a redução do
número de ciclos usando o código em Assembly varia de $64,5\%$ a
$94,4\%$. A partir de $100.000$ pontos, a redução dos valores de uso de ciclos
do processador cai para menos de $40\%$.

\section*{DISCUSSÃO}

Quando se tem por objetivo melhorar a performance dos programas que
processam grande quantidade de dados, caso comum em processamento de
imagens e sinais, o conhecimento do hardware utilizado para execução
das tarefas pode contribuir para identificar pontos de otimização.

As linguagens de alto nível como o C fornecem um nível de abstração e
portabilidade que contribuem para facilitar a codificação com maior
facilidade de gerenciamento do código fonte. Porém, alguns recursos
recentes do processadores modernos ainda não estão disponíveis nas
linguagens de programação mais comuns. Por este motivo, a integração
de código de alto nível com Assembly, usando esta última em pontos que
exijam maior performance pode ser uma solução para reduzir o tempo de
processamento total.

O processador Intel\TR{} Core\TM{} i7 possui registradores de 256 bits
para a realização das operações em múltiplos dados, através de sua
extensão AVX ({\em Advanced Vector Extensions}), possuindo operações
específicas para dados multimídia.  Outros pontos de otimização mais
ligados ao sistema operacional, como por exemplo melhor controle do
sistema de entrada/saída (E/S), também cooperam para a redução do
tempo total de processamento. Por exemplo, a manipulação do buffer,
uso de operações de E/S não bloqueante e mapeamento do arquivo
contendo os dados para memória principal podem reduzir o tempo de
acesso aos dados, aumentando o fluxo de instruções no processador.

\section*{CONCLUSÃO}

O uso da extensão SSE do processador Intel\TR{} para o cálculo da
correlação cruzada entre duas séries de valores inteiros, de diferente
tamanhos e com a geração do número inteiro baseada no índice do
elemento, reduziu o número de ciclos utilizados no processador pelos
programas em média $38,37\%$, número este que pode ser maior para
computadores com alta capacidade de memória cache.

\section*{REFERÊNCIAS}

\begin{thebibliography}{9}

\bibitem{blum} BLUM, Richard. Professional Assembly Language.  Editora Wrox, 2005.

\bibitem{intel} INTEL\TR{} Corporation. Intel\TR{} 64 and IA-32 Architectures Software
Developer’s Manual. Volume 2B, 2012.

\bibitem{ray} SEYFARTH, Ray. Introduction to 64 Bit Intel Assembly: Language
Programming for Linux, 2012 (Edição Kindle).

\end{thebibliography}

\end{document}
